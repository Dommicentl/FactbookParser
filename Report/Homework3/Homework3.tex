\documentclass{article}
\usepackage{fullpage}
\usepackage{biblatex}
\usepackage{graphicx}
\usepackage{listings}
\bibliography{refs.bib}
\setlength{\parskip}{4pt}
\setlength{\parindent}{0pt}
\title{Advanced databases: Homework 3}
\author{Leendert Dommicent and Jorn Van Loock}
\begin{document}
\maketitle
\section{Introduction}
In this report we describe the actions that we have taken to complete the third homework. In section \ref{sec:dbpedia} we describe our problems with DBpedia and the reason why we decided not to use it. In section \ref{sec:startdb} we give the database that we actually used and which data we parsed. Finally in section \ref{sec:conclusion} we give our conclusion of this homework.
\section{The DBpedia problem}
\label{sec:dbpedia}
DBpedia\cite{dbpedia} is in a way the owl version of wikipedia. In wikipedia we found a page with a list of terrorist organizations. We then browsed to the corresponding DBpedia webpage. It however didn't contain any relation to the terrorist organizations that were present in the wikipedia version.\par
Next we tried to search for these organizations manually and we found a number of them but they all had different very specific classes like \textit{libanonmilitia}. Also they didn't had a parent class in common. This makes it very hard to use the SPARQL interface because we can't query every element from a certain class.\par
The only option was to go for every terrorist organization of the wikipedia page to the corresponding DBpedia page and parse the data. This would be a lot of work because also the data wasn't easy to parse, you didn't always have a religion property for every organization etc. We then decided to search for other databases with interesting information.
\section{Terrorist Organization Profiles database}
\label{sec:startdb}
We finally found a relative good data source namely the Terrorist Organization Profiles database\cite{start}. This database is from the same organization as the database with the terrorist attacks. This is good because they use the same name conventions and notation as the attack database, which we already parsed. We will have a lot less doubles in our ontology because of this.\par
A downside was that the ontology didn't contain religions but classifications like religious, separatists, anarchists, etc. We think however that this is perhaps more useful because not all terrorist groups have a religion. We decided to parse this and to add it to the database. This could result in relations like anarchists are more violent are take more hostage than religious groups, which can be interesting. The database also contains the countries where the organizations operate, which we also added. We thought that also this could maybe be interesting.\par
We did the parsing again with the JSoup\cite{jsoup} library en the writing to the owl file with the OwlAPI\cite{owlapi}. You can find the used code in our GitHub repository\cite{githubproject}.
\section{Conclusion}
So we described in this report that we didn't use DBpedia because of the problems with the classes. We described which, in our opinion better database we used namely the Terrorist Organization Profiles database. Finally we explained why we parsed the classification and bases of operation relations.\par
We now have a database with more interesting information than we had in homework 2.
\label{sec:conclusion}
\printbibliography
\end{document}
