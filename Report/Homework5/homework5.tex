\documentclass{article}
\usepackage{fullpage}
\usepackage{listings}
\usepackage{graphicx}
\usepackage{subcaption}
\usepackage{amsmath}
\usepackage{amsxtra}
\title{Advanced databases: Homework 5}
\author{Leendert Dommicent and Jorn Van Loock}
\begin{document}
\maketitle
\setlength{\parskip}{4pt}
\setlength{\parindent}{0pt}
\section{Introduction}
In this homework we are going to try classifying stories in two classes, the ones who contain a princess and the ones who don't. We will do this by writing a learner and classifier in Hadoop which uses the na\"ive Bayes algorithm. As features for this algorithm we will take the occurrences of the words, \textit{castle}, \textit{king} and \textit{prince}. So this features are binary it is either yes or no.
\section{The learner}
We have written a Java program who classified 40 stories by looking if there is a word princess in it. He puts the two classes in two folders \textit{true} and \textit{false}. For this two folders we have to write a program in Hadoop which counts how many files contains each feature.
\subsection{Hadoop learner}
Our Hadoop program has two map reduce steps. The first map step will read the files and every time it finds a feature it sends \textit{$<$Word@file,\_$>$}. With every map step he also sends \textit{$<$Count@file,\_$>$} to find the total files. Then the reducer reduces this so every key occurs only once. Now we have a list with which of the features are in which file and lines with \textit{Count} in the beginning for every file.\par
This list goes to the second mapper. This mapper splits every entry on the @ sign. For every entry it handles, it sends \textit{$<$Word,1$>$}. Now every reducers receives the occurrences of one feature and counts these. There is also a reducer who receives the \textit{Count} keys who will then calculate the total number of files.\par
We ran this Hadoop program on the princess stories and the other stories. This were our results:
\begin{figure}[!ht]
\centering
\begin{subfigure}{.5\textwidth}
\centering
\begin{tabular}{c}
\begin{lstlisting}
Total	12
castle	6
king 	11
prince	6
\end{lstlisting}
\end{tabular}
\label{fig:results_princes}
\caption{The results of the princess stories}
\end{subfigure}
\begin{subfigure}{.4\textwidth}
\centering
\begin{tabular}{c}
\begin{lstlisting}
Total	38
castle	6
king	8
prince	1
\end{lstlisting}
\end{tabular}
\label{fig:results_other}
\caption{The results of other stories}
\end{subfigure}
\end{figure}
With these results we can calculate the following chances:
\begin{gather*}
P(Castle | Princess) = 6/12 = 0.5\\
P(King | Princess) = 11/12 \approx 0.91667\\
P(Prince | Princess) = 6/12 = 0.5\\ \\
P(Castle | \sim Princess) = 6/38 \approx 0.15789\\
P(King | \sim Princess) = 8/38 \approx 0.2105\\
P(Prince | \sim Princess) = 1/38 \approx 0.02631\\ \\
\end{gather*}
We will need this values in our classifier.
\section{Classifier}
\end{document}
