\documentclass[a4]{article}
\usepackage{fullpage}
\usepackage{biblatex}
\usepackage{listings}
\usepackage{listings}
\usepackage{placeins}
\lstset{captionpos=b}
\bibliography{refs.bib}
\title{Advanced database systems: Homework 6}
\author{Leendert Dommicent and Jorn Van Loock}
\date{}
\begin{document}
\maketitle
\setlength{\parindent}{0px}
\setlength{\parskip}{8px}
\section{Introduction}
This document describes what we have don for the final homework. In section \ref{sec:problem} we will describe the problem domain and why the data mining project that we have carried out can be usefull. In section \ref{sec:gathering_data} we will explain where we got the data from and why we chose these data sources. Next in section \ref{sec:preparing_data} we explain how the data is prepared for use with Weka. Which data do we drop and why, which data we combine and so forth. In section \ref{sec:weka} we will describe how we will use Weka to find interesting results in the data. Section \ref{sec:results} will explain the results that we have and what they imply. Finally in section \ref{sec:conclusion} we will formulate a conclusion about this homework.
\section{The problem domain}
\label{sec:problem}
Almost every day the world suffers under terrorist attacks. Those attacks are carried out by individuals in their own name or in the name of a bigger organisation. New terrorist organizations start when other stop. For these new organizations the whole investigation about their actions must be done all over again. Wouldn't it be great if we could learn from the existing and past organizations and apply this on the newly created organizations? In a very minimal way, this is after all a homework and not a thesis, we will try to do this.\par
All terrorist attacks can be classified by a type, for example bombing or hostage taking. They also target a specific victim type like the government or a private company. We can also classify the different types of terrorist organizations. You have organizations with a cultural background and others with a political.\par
In this research we want to determine if there is a connection between the type of terrorist organization, the continent they are active in and the type of attacks they are doing. Wouldn't it be interesting if a new terrorist organization rises we can predict which attack type they will use or which people they are most likely to attack? We are very confident that this is a very interesting and useful domain to do research in.
\section{Gathering the right data}
\label{sec:gathering_data}
To predict this we first have to acquire the necessary data of the terrorist organizations we know. We need to know the types of organizations, the attacks they do and where they do it.\par
For the attacks we used GTD (Global Terrorist Database)\cite{gtd}. This database contains terrorist attacks all over the world. It contains the countries where they were executed, the type of the attacks and the type of the victims. We wrote a Java parser that parsed all these attacks and added them to an OWL ontology. You can find more information in this\cite{homework2} report.\par To know in which continent the countries of the attacks are located we used the CIA World Factbook\cite{factbook}. We also wrote a Java parser for this database and added the content to our owl ontology. Again you can find more information in this\cite{homework2} report.\par
We also needed more information about the organizations themselves. For this we used the Terrorist Organization Profiles database\cite{start}. This database is from the same makers as GTD so names from the two databases should match. Because of this we do not have double individuals with a slightly different name in our OWL database. So also for this database we wrote a Java parser to add the data to our ontology. More information of this you can find in this\cite{homework3} report.\par
We now have all the data we need in an OWL ontology. This ontology file also has an accompanying readme file. In this file you can find how the ontology itself is build up. Which classes and what sort of individuals it contains. It also explains which relations (object and data) are in the ontology and for which individuals they are filled in. In the next section we will describe how we will transform this data to be able to use Weka.
\section{Preparing the data}
\label{sec:preparing_data}
We need to convert our OWL ontology to an ARFF file that can be read by Weka. On the way we will simplify the data a little bit. We will write this converting program in Java. We need a way to read out our OWL ontology. For this we used the OWLAPI\cite{owlapi}. This is a powerful API for Java that allows you to read and write data to and from your ontology. We already used it to write and populate our ontologies decsribed in the previous section but will now use it to read certain elements from the ontology.\par
We have three different ways that we prepared the data. We will explain them all. The results of the data mining depends on the way we prepared the dataThis means that also the section \ref{sec:results} is devided in three parts, one for every way we prepared the data.\par
\subsection{1 entry for every terrorist organization}
\label{sec:1_for_organization}
To begin the first conversion we will search all attack individuals from our ontology. For every individual we will first get the terrorist organization that is responsible for the attack. If we didn't already came across this organization we will create a new organization object otherwise we will use the existing object. We will then add the country of the attack and his victim type and attack type to this organization object. If any information is missing we will just ignore the current attack. After this step is completed we have an object for every terrorist organization. This object contains all the victim types and attack types they have used, the amount of each of the types and the countries in which they operate. Now we only need the type of organization of the organization. So we search for this information in the OWL ontology and add this to the object. Again when we do not find the type in the ontology we drop the organization, because it just isn't useful for learning.\par
In the next step we have to simplify the data in de objects and write it to an ARFF file. First we are going to simplify the countries of the organization object. For every country in the object we will search the continent of that country. Then we will count which continent is the most occurring and throw away the other continents. We do the same for the attack types and the victim types, so only the most occurring are kept. We do understand that we make a huge generalisation by simplifying the data like this.\par
Now we have to convert all this organization objects to an ARFF file. First we have to create the heading of the ARFF file. To do this we have to identify the attributes of the problem and the instances. The instances are clearly the terrorist organizations. The attributes are the name of the organization, the type of the organization and the most used victim type and attack type. The first attribute is a simple string and isn't used in the actual data mining. It is just used to keep the ARFF file more clear. The other type are nominal. This means that we have to declare all the possible values up front in the header of the file. To do this we retrieve all the possible continents, attack types and victim types and add them to the header of the ARFF file. You can find the resulting header in figure \ref{fig:arff_header}. Notice that all the spaces are replaced by underscores. This is because we noticed that Weka gives errors when there are spaces in the content of attributes. So we replace spaces by underscores in the entire ARFF file. Also the (') sign gave problems so we just deleted them in the entire file.\par
In figure \ref{fig:arff_header} you can see which possible values we use. You can see that we do not just use continents but that we sort of split up the large continents. In total we have 10 possible continent values. We hope that this split up gives us better result instead of just using the normal continents. We also thought about using countries instead of continents, but we are afraid that this gives us results that are too specific and are by consequence not very useful.\par
\begin{figure}[!h]
\centering
\begin{tabular}{c}
\begin{lstlisting}
@Relation titanic
@ATTRIBUTE name string
@ATTRIBUTE classification {Anarchist,Anti-Globalization,Communist/Socialist,
Environmental,Leftist,Nationalist/Separatist,Racist,Religious,
Right-Wing_Conservative,Right-Wing_Reactionary,n/a,Other}
@ATTRIBUTE continent {Africa,Australia-Oceania,Central_America_and_Caribbean,
Central_Asia,East_&_Southeast_Asia,Europe,Middle_East,North_America,
South_America,South_Asia}
@ATTRIBUTE victimType {Abortion_Related,Airports_&_Airlines,Business,
Educational_Institution,Food_or_Water_Supply,Government_(Diplomatic),
Government_(General),Journalists_&_Media,Maritime,Military,NGO,Other,Police,
Private_Citizens_&_Property,Religious_Figures/Institutions,Telecommunication,
Terrorists,Tourists,Transportation,Unknown,Utilities,Violent_Political_Party}
@ATTRIBUTE attackType {Armed_Assault,Assassination,Bombing/Explosion,
Facility/Infrastructure_Attack,Hijacking,Hostage_Taking_(Barricade_Incident),
Hostage_Taking_(Kidnapping),Unarmed_Assault,Unknown}
@DATA
\end{lstlisting}
\end{tabular}
\caption{The header of the created ARFF file}
\label{fig:arff_header}
\end{figure}
\par
Now that we have the header we have to convert every object to an instance and by consequence a line in the ARFF file. You can find the structure of the line in figure \ref{fig:entry}. Notice that they are just the properties of the organisation object separated by a comma.\par
\begin{figure}[!h]
\centering
\begin{tabular}{c}
\begin{lstlisting}
name_organization,type_organization,continent,victim_type,attack_type
\end{lstlisting}
\end{tabular}
\caption{An instance entry in the ARFF file}
\label{fig:entry}
\end{figure}
\subsection{1 entry for every attack}
\label{sec:1_for_attack}
Maybe the simplification of all victim types and attack types from every attack to one for every terrorist organization is too hard. Maybe it's a better solution to give weka this information for every attack. To do this we have to change our precious Java program.\par
The Java program from section \ref{sec:1_for_organization} creates one object for every terrorist organization. If it iterates over an attack of an organization for which the terrorist organization object is already initialized, it will just return this object to work with. It will then for example add another attack type to this object. To make the requested modification we just have to create a new terrorist organization object for every attack in the database. In the rest of this section we will call this object an entry object to keep things clear. So we just remove the code that checks if the entry object is already created. The rest of the code that simplifies the entry objects is now unnecessary but we can leave it there. Because every entry object now has only one victim type and one attack type, because an attack has only one victim type and one attack type, it is automatically selected by the simplification algorithm.\par
The structure of the resulting ARFF file is the same as in section \ref{sec:1_for_organization}. However in section  \ref{sec:1_for_organization} we have a line like the one in figure \ref{fig:entry} for every organization. Now however we have such a line for every attack but with the same structure. So for example if the IRA has carried out multiple attacks they will have multiple entries in the file. So with this way of preperation we have a much bigger file than the first way. In the file of \ref{sec:1_for_organization} we have around 200 entries, with this system we have around 13000 entries. So with this method we have a lot more data to work with.
\subsection{1 entry for every attack with neighbours}
While we were explaining our plans for this homework, it was brought to our attention that continent is maybe a too general feature. Maybe it was better to work with regions instead of full continents even when we split some big continents up. Because of the time constraints of this project it wasn't feasible to manually enter a region for every country. We also didn't find a good data source that contained this data. So we had to come up with another solution that made us able to split up the continents in smaller pieces and which could be done in an automated fashion.\par
We found a solution that would chain neighbouring  countries together. So for every attack we fetch the country and add it to the entry object. After we do this we traverse all created entry objects. For every object we get the neighbouring countries of the country contained by the object and add it to the object. We then delete duplicate countries. If we want to we can do this 2 steps several times to make the regions bigger. For this homework however we will not do this but the code can be extended easily to support this. We now have entry objects that contain the country of the attack and the neighbouring countries. We now have to translate this to an ARFF file.\par
To translate this we will use the structure you can find in table \ref{fig:structure_arrf_3}. So we kept all the features from the previous methods but we added one for every country in the database. This feature will contain zero or one depending of the fact if the attack took place in this country or in a neighbouring country.\par
\begin{table}[!h]
\centering
\begin{tabular}{|c|c|c|c|c|c|c|c|c|}\hline
Name & Type & Continent & VictimType & AttackType & Afghanistan & Albania & ... & Zimbabwe \\ \hline
\end{tabular}
\caption{Structure of the ARFF file with the third method}
\label{fig:structure_arrf_3}
\end{table}
We now need to expend the functionality of our Java program to fill in the structure. We added a method who created a string for every entry object with comma separated zero's and one's depending on which countries the entry object contains. We then concatenate this string with the old string. You can find the result of an existing entry in table \ref{fig:example_arrf_3}. In the ARFF file this data is of course comma separated like the other methods.\par
\begin{table}[!h]
\centering
\begin{tabular}{|c|c|c|c|c|c|c|c|c|}\hline
Name & Type & Continent & VictimType & AttackType & Afghanistan & Albania & ... & Zimbabwe \\ \hline
JRA & Communist & Asia & Airports & Hijacking & 0 & 1 & ... & 0 \\ \hline
\end{tabular} 
\caption{Example of a fictive entry}
\label{fig:example_arrf_3}
\end{table}
The final thing we have to do is adapt the ARFF header of the previous methods to accommodate the new features. To do this we wrote a method in our Java program that gets all countries inside the ontology. For every country found it will write the line in figure \ref{fig:country_header} to the header of the ARFF file. All this entries make the header much bigger than before. You can see that it is a nominal feature that can contain the values zero and one like previously explained.
\begin{figure}[!h]
\centering
\begin{tabular}{c}
\begin{lstlisting}
@ATTRIBUTE name_country {0,1}
\end{lstlisting}
\end{tabular}
\caption{A country entry within the ARFF header}
\label{fig:country_header}
\end{figure}
To make the whole ARFF file a little bit smaller we deleted all country columns who contained only zero's and adapted the ARFF header accordingly.
\subsection{The resulting Java program}
You can find the resulting program in our GitHub repository\cite{githubproject} which is publicly available. The class \textit{FactbooktoARFF} in package \textit{com.conbit.factbookparser.convertors} handles the actual conversion from OWL to ARFF. It contains booleans to determine which method of the three we explained you want to use.
\section{Using Weka for data mining}
\label{sec:weka}
To be able to predict the attack types and victim types of future terrorist organizations we need to learn some rules about them. We can also try to find a way to classify the current organizations and if a new organization comes up we can classify them and by result know their preferred attack and victim type. Before starting the mining algorithm on the data we first have to remove the feature name because it isn't relevant, it was just there for debugging purposes. This can easily be done in Weka.\par
First we will try to find certain rules in the data. An example of such a rule could be: \textit{Religious organizations that operate in Europe mostly use hostage tacking as their attack type.} For this rule mining we will try to use the rule miner Apriori. Apriori will try to find rules between a certain metric. You can for example state that it has to find rules with a confidence higher then 0.9 for example. We will use Weka with the standard parameters only the minimum confidence is changed to 0.5 to get more result. With these parameters Apriori will try to find rules with a certain support it will then lower the support and try again. He will do this until he has found 10 rules. The minimum support is 0.1 with the default parameters. If we have changed the minimum support for a certain test we will explicitly mention it.\par
Next we will try to classify the different organizations using a decision tree. To build this tree we are planning to use the J48 algorithm of Weka. This algorithm can build such a decision tree. With this tree we can than classify new organizations and identify which attack and or victim types they will try to use. We will run the J48 algorithm again with the standard parameters. We will always build 2 trees, one for prediction the victim type and one for the attack type.\par
\section{Results}
\label{sec:results}
In this section we will give the results from Weka for the three data files, one for each preparing method. We will also try to interpret this message. For every preparing method we will give the results of both Apriori and J48.
\subsection{Results for datafile with 1 entry for every organization}
\subsubsection{Apriori}
\FloatBarrier
\begin{figure}[!h]
\centering
\begin{tabular}{c}
\begin{lstlisting}
1. victimType=Business 51 ==> attackType=Bombing/Explosion 39    conf:(0.76)
2. continent=Middle_East 41 ==> attackType=Bombing/Explosion 26    conf:(0.63)
3. continent=Europe 67 ==> attackType=Bombing/Explosion 42    conf:(0.63)
4. classification=Other 46 ==> attackType=Bombing/Explosion 28    conf:(0.61)
5. classification=Religious 55 ==> attackType=Bombing/Explosion 31    conf:(0.56)
6. victimType=Business 51 ==> continent=Europe 27    conf:(0.53)
7. classification=Nationalist/Separatist 53 ==> attackType=Bombing/Explosion 28    conf:(0.53)
\end{lstlisting}
\end{tabular}
\caption{Apriori results for first data file}
\label{fig:apriori_1}
\end{figure}
In figure \ref{fig:apriori_1} you can find the result of the Apriori data mining in Weka. At first sight it seems that it generated some interesting rules. However if we look closer we see that it almost always predicts the attack type \textit{Bombing/Explosion}. Rule 5 is interesting because it tells us that most businesses that have been attacked are in Europe. However this rule does not have a large confidence.\par
We changed to minimum support to 0.05 for a second test which we think is pretty low in a dataset of 231 instances. However Weka only generated more rules with a prediction of \textit{Bombing/Explosion} as attack type. So the only information we in fact really got is that most terrorist attacks are bombings.
\subsubsection{J48}
The tree for the attack type only has one leaf \textit{Bombing/Explosion}. The three classifies 55\% correctly according to 10-fold cross-validation. This confirms the rules of Apriori. The most attacks that are being executed are bombing attacks.\par
The tree for the victim types only classified 27\% correctly. It however did have 1 interesting branches. Bombing attacks in Europe mainly focus on business targets, this is true in 22 of the 42 occurrences. This was the branch with the best confidence and a decent support. And still the confidence isn't high. This resulting tree is quite bad, so that maybe explains why Apriori couldn't give any rules.
\subsubsection{Conclusion}
It's clear that we simplified the data to much. For the attack types we got a decent result but it wasn't a very helpful result. For the victim types we didn't get a good result at all. We hope we get better results for the other data files.
\section{Limitations}
In this section we will give the results for the datafile constructed in section \ref{sec:1_for_organization}.
\label{sec:limitations}
\section{Conclusion}
\label{sec:conclusion}
In this document we tried to clearly explain what our plans were for the final homework, homework 6. We tried to structure this document according to the Crisp-DM model. It also contains which steps we already completed and which steps we still have to do.\par
Because this is a very elaborate and through explanation of what we have done already and are planning to do, it is very likely we will use parts of this document in the final report. As always all our code that we used can be found on our GitHub project\cite{githubproject}.
\printbibliography
\end{document}
